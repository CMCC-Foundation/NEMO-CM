%% =================================================================================================
%% Common abstract
%% =================================================================================================

``Nucleus for European Modelling of the Ocean'' as \NEMO\ is a state-of-the-art modelling framework of
ocean-related engines for research activities and forecasting services in oceanography and climatology,
developed in a sustainable way since 2008 by a European consortium of 5 institutes
(\CMCC{CMCC} | \CNRS{CNRS} | \MOI{Mercator Océan} | \UKMO{Met Office} | \NERC{NERC}).
It is intended to be a flexible tool for studying the physical and biogeochemical phenomena in
the ocean circulation, as well as its interactions with the components of the Earth climate system,
over a wide range of space and time scales.

Concerning the physics, the fundamental engine for the \textcolor{blue}{``blue ocean''} solves
the primitive equations of the ocean \{thermo\}dynamics.
It can be supplemented by the \textcolor{gray}{``white ocean''} for sea-ice \{thermo\}dynamics,
brine inclusions and subgrid-scale thickness variations (\SIcube), and also by
the \textcolor{green}{``green ocean''} for \{on,off\}line oceanic tracers transport and
biogeochemical processes (\TOP-\PISCES).
External alternative models can be used instead of the core engines (\eg \BFM).
Regarding the numerics, main features include versatile data assimilation interface,
agile diagnostics generation thanks to \XIOS\ software,
ocean-atmosphere coupling via the \OASIS\ library, and
seamless embedded zooms with the \AGRIF\ 2-way nesting package.
