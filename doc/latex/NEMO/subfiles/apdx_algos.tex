\documentclass[../main/NEMO_manual]{subfiles}

\begin{document}

\chapter{Note on some algorithms}
\label{apdx:ALGOS}

\chaptertoc

\paragraph{Changes record} ~\\

{\footnotesize
  \begin{tabularx}{\textwidth}{l||X|X}
    Release & Author(s) & Modifications \\
    \hline
    {\em   4.0} & {\em ...} & {\em ...} \\
    {\em   3.6} & {\em ...} & {\em ...} \\
    {\em   3.4} & {\em ...} & {\em ...} \\
    {\em <=3.4} & {\em ...} & {\em ...}
  \end{tabularx}
}

\clearpage

This appendix some on going consideration on algorithms used or planned to be used in \NEMO.

%% =================================================================================================
\section{Upstream Biased Scheme (UBS) (\protect\np[=.true.]{ln_traadv_ubs}{ln\_traadv\_ubs})}
\label{sec:ALGOS_tra_adv_ubs}

The UBS advection scheme is an upstream biased third order scheme based on
an upstream-biased parabolic interpolation.
It is also known as Cell Averaged QUICK scheme (Quadratic Upstream Interpolation for Convective Kinematics).
For example, in the $i$-direction:
\begin{equation}
  \label{eq:ALGOS_tra_adv_ubs2}
  \tau_u^{ubs} = \left\{
	 \begin{aligned}
      & \tau_u^{cen4} + \frac{1}{12} \,\tau"_i	   & \quad \text{if }\ u_{i+1/2} \geqslant 0 \\
      & \tau_u^{cen4} - \frac{1}{12} \,\tau"_{i+1} & \quad \text{if }\ u_{i+1/2}       <       0
    \end{aligned}
  \right.
\end{equation}
or equivalently, the advective flux is
\begin{equation}
  \label{eq:ALGOS_tra_adv_ubs2}
  U_{i+1/2} \ \tau_u^{ubs}
  =U_{i+1/2} \ \overline{ T_i - \frac{1}{6}\,\tau"_i }^{\,i+1/2}
  - \frac{1}{2}\, |U|_{i+1/2} \;\frac{1}{6} \;\delta_{i+1/2}[\tau"_i]
\end{equation}
where $U_{i+1/2} = e_{1u}\,e_{3u}\,u_{i+1/2}$ and
$\tau "_i =\delta_i \left[ {\delta_{i+1/2} \left[ \tau \right]} \right]$.
By choosing this expression for $\tau "$ we consider a fourth order approximation of $\partial_i^2$ with
a constant i-grid spacing ($\Delta i=1$).

Alternative choice: introduce the scale factors:
$\tau "_i =\frac{e_{1T}}{e_{2T}\,e_{3T}}\delta_i \left[ \frac{e_{2u} e_{3u} }{e_{1u} }\delta_{i+1/2}[\tau] \right]$.

This results in a dissipatively dominant (\ie\ hyper-diffusive) truncation error
\citep{shchepetkin.mcwilliams_OM05}.
The overall performance of the advection scheme is similar to that reported in \cite{farrow.stevens_JPO95}.
It is a relatively good compromise between accuracy and smoothness.
It is not a \emph{positive} scheme meaning false extrema are permitted but
the amplitude of such are significantly reduced over the centred second order method.
Nevertheless it is not recommended to apply it to a passive tracer that requires positivity.

The intrinsic diffusion of UBS makes its use risky in the vertical direction where
the control of artificial diapycnal fluxes is of paramount importance.
It has therefore been preferred to evaluate the vertical flux using the TVD scheme when
\np[=.true.]{ln_traadv_ubs}{ln\_traadv\_ubs}.

For stability reasons, in \autoref{eq:TRA_adv_ubs}, the first term which corresponds to
a second order centred scheme is evaluated using the \textit{now} velocity (centred in time) while
the second term which is the diffusive part of the scheme, is evaluated using the \textit{before} velocity
(forward in time).
This is discussed by \citet{webb.de-cuevas.ea_JAOT98} in the context of the Quick advection scheme.
UBS and QUICK schemes only differ by one coefficient.
Substituting 1/6 with 1/8 in (\autoref{eq:TRA_adv_ubs}) leads to the QUICK advection scheme \citep{webb.de-cuevas.ea_JAOT98}.
This option is not available through a namelist parameter, since the 1/6 coefficient is hard coded.
Nevertheless it is quite easy to make the substitution in \mdl{traadv\_ubs} module and obtain a QUICK scheme.

NB 1: When a high vertical resolution $O(1m)$ is used, the model stability can be controlled by vertical advection
(not vertical diffusion which is usually solved using an implicit scheme).
Computer time can be saved by using a time-splitting technique on vertical advection.
This possibility have been implemented and validated in ORCA05-L301.
It is not currently offered in the current reference version.

NB 2: In a forthcoming release four options will be proposed for the vertical component used in the UBS scheme.
$\tau_w^{ubs}$ will be evaluated using either \textit{(a)} a centered $2^{nd}$ order scheme,
or \textit{(b)} a TVD scheme, or \textit{(c)} an interpolation based on conservative parabolic splines following
\citet{shchepetkin.mcwilliams_OM05} implementation of UBS in ROMS, or \textit{(d)} an UBS.
The $3^{rd}$ case has dispersion properties similar to an eight-order accurate conventional scheme.

NB 3: It is straight forward to rewrite \autoref{eq:TRA_adv_ubs} as follows:
\begin{equation}
  \label{eq:ALGOS_tra_adv_ubs2}
  \tau_u^{ubs} = \left\{
	 \begin{aligned}
      & \tau_u^{cen4} + \frac{1}{12} \tau"_i		& \quad \text{if }\ u_{i+1/2} \geqslant 0 \\
      & \tau_u^{cen4} - \frac{1}{12} \tau"_{i+1}	& \quad \text{if }\ u_{i+1/2}       <       0
    \end{aligned}
  \right.
\end{equation}
or equivalently
\begin{equation}
  \label{eq:ALGOS_tra_adv_ubs2}
  \begin{split}
    e_{2u} e_{3u}\,u_{i+1/2} \ \tau_u^{ubs}
    &= e_{2u} e_{3u}\,u_{i+1/2} \ \overline{ T - \frac{1}{6}\,\tau"_i }^{\,i+1/2} \\
    & - \frac{1}{2} e_{2u} e_{3u}\,|u|_{i+1/2} \;\frac{1}{6} \;\delta_{i+1/2}[\tau"_i]
  \end{split}
\end{equation}
\autoref{eq:TRA_adv_ubs2} has several advantages.
First it clearly evidences that the UBS scheme is based on the fourth order scheme to which
is added an upstream biased diffusive term.
Second, this emphasises that the $4^{th}$ order part have to be evaluated at \emph{now} time step,
not only the $2^{th}$ order part as stated above using \autoref{eq:TRA_adv_ubs}.
Third, the diffusive term is in fact a biharmonic operator with a eddy coefficient which
is simply proportional to the velocity.

laplacian diffusion:
\begin{equation}
  \label{eq:ALGOS_tra_ldf_lap}
  \begin{split}
    D_T^{lT} =\frac{1}{e_{1T} \; e_{2T}\;  e_{3T} } &\left[ {\quad \delta_i
        \left[ {A_u^{lT} \frac{e_{2u} e_{3u} }{e_{1u} }\;\delta_{i+1/2}
            \left[ T \right]} \right]} \right. \\
    &\ \left. {+\; \delta_j \left[
          {A_v^{lT} \left( {\frac{e_{1v} e_{3v} }{e_{2v} }\;\delta_{j+1/2} \left[ T
                \right]} \right)} \right]\quad } \right]
  \end{split}
\end{equation}

bilaplacian:
\begin{equation}
  \label{eq:ALGOS_tra_ldf_lap}
  \begin{split}
    D_T^{lT} =&-\frac{1}{e_{1T} \; e_{2T}\;  e_{3T}} \\
    & \delta_i \left[  \sqrt{A_u^{lT}}\ \frac{e_{2u}\,e_{3u}}{e_{1u}}\;\delta_{i+1/2}
      \left[ \frac{1}{e_{1T}\,e_{2T}\, e_{3T}}
        \delta_i \left[ \sqrt{A_u^{lT}}\ \frac{e_{2u}\,e_{3u}}{e_{1u}}\;\delta_{i+1/2}
          [T] \right] \right] \right]
  \end{split}
\end{equation}
with ${A_u^{lT}}^2 = \frac{1}{12} {e_{1u}}^3\ |u|$,
\ie\ $A_u^{lT} = \frac{1}{\sqrt{12}} \,e_{1u}\ \sqrt{ e_{1u}\,|u|\,}$
it comes:
\begin{equation}
  \label{eq:ALGOS_tra_ldf_lap}
  \begin{split}
    D_T^{lT} =&-\frac{1}{12}\,\frac{1}{e_{1T} \; e_{2T}\;  e_{3T}} \\
    & \delta_i \left[ e_{2u}\,e_{3u}\,\sqrt{ e_{1u}\,|u|\,}\;\delta_{i+1/2}
      \left[ \frac{1}{e_{1T}\,e_{2T}\, e_{3T}}
        \delta_i \left[ e_{2u}\,e_{3u}\,\sqrt{ e_{1u}\,|u|\,}\;\delta_{i+1/2}
          [T] \right] \right] \right]
  \end{split}
\end{equation}
if the velocity is uniform (\ie\ $|u|=cst$) then the diffusive flux is
\begin{equation}
  \label{eq:ALGOS_tra_ldf_lap}
  \begin{split}
    F_u^{lT} = - \frac{1}{12}
    e_{2u}\,e_{3u}\,|u| \;\sqrt{ e_{1u}}\,\delta_{i+1/2}
    \left[ \frac{1}{e_{1T}\,e_{2T}\, e_{3T}}
      \delta_i \left[ e_{2u}\,e_{3u}\,\sqrt{ e_{1u}}\:\delta_{i+1/2}
        [T] \right] \right]
  \end{split}
\end{equation}
beurk....  reverte the logic: starting from the diffusive part of the advective flux it comes:

\begin{equation}
  \label{eq:ALGOS_tra_adv_ubs2}
  \begin{split}
    F_u^{lT} &= - \frac{1}{2} e_{2u} e_{3u}\,|u|_{i+1/2} \;\frac{1}{6} \;\delta_{i+1/2}[\tau"_i]
  \end{split}
\end{equation}
if the velocity is uniform (\ie\ $|u|=cst$) and
choosing $\tau "_i =\frac{e_{1T}}{e_{2T}\,e_{3T}}\delta_i \left[ \frac{e_{2u} e_{3u} }{e_{1u} } \delta_{i+1/2}[\tau] \right]$

sol 1 coefficient at T-point ( add $e_{1u}$ and $e_{1T}$ on both side of first $\delta$):
\begin{equation}
  \label{eq:ALGOS_tra_adv_ubs2}
  \begin{split}
    F_u^{lT} &= - \frac{1}{12} \frac{e_{2u} e_{3u}}{e_{1u}}\;\delta_{i+1/2}\left[ \frac{e_{1T}^3\,|u|}{e_{1T}e_{2T}\,e_{3T}}\,\delta_i \left[ \frac{e_{2u} e_{3u} }{e_{1u} } \delta_{i+1/2}[\tau] \right] \right]
  \end{split}
\end{equation}
which leads to ${A_T^{lT}}^2 = \frac{1}{12} {e_{1T}}^3\ \overline{|u|}^{\,i+1/2}$

sol 2 coefficient at u-point: split $|u|$ into $\sqrt{|u|}$ and $e_{1T}$ into $\sqrt{e_{1u}}$
\begin{equation}
  \label{eq:ALGOS_tra_adv_ubs2}
  \begin{split}
    F_u^{lT} &= - \frac{1}{12} {e_{1u}}^1 \sqrt{e_{1u}|u|} \frac{e_{2u} e_{3u}}{e_{1u}}\;\delta_{i+1/2}\left[ \frac{1}{e_{2T}\,e_{3T}}\,\delta_i \left[ \sqrt{e_{1u}|u|} \frac{e_{2u} e_{3u} }{e_{1u} } \delta_{i+1/2}[\tau] \right] \right] \\
    &= - \frac{1}{12} e_{1u} \sqrt{e_{1u}|u|\,} \frac{e_{2u} e_{3u}}{e_{1u}}\;\delta_{i+1/2}\left[ \frac{1}{e_{1T}\,e_{2T}\,e_{3T}}\,\delta_i \left[ e_{1u} \sqrt{e_{1u}|u|\,} \frac{e_{2u} e_{3u} }{e_{1u}} \delta_{i+1/2}[\tau] \right] \right]
  \end{split}
\end{equation}
which leads to ${A_u^{lT}} = \frac{1}{12} {e_{1u}}^3\ |u|$

%% =================================================================================================
\section{Leapfrog energetic}
\label{sec:ALGOS_LF}

We adopt the following semi-discrete notation for time derivative.
Given the values of a variable $q$ at successive time step,
the time derivation and averaging operators at the mid time step are:
\[
  % \label{eq:ALGOS_dt_mt}
  \begin{split}
    \delta_{t+\rdt/2} [q]     &=  \  \ \,   q^{t+\rdt}  - q^{t}		\\
    \overline q^{\,t+\rdt/2} &= \left\{ q^{t+\rdt} + q^{t} \right\} \; / \; 2
  \end{split}
\]
As for space operator,
the adjoint of the derivation and averaging time operators are $\delta_t^*=\delta_{t+\rdt/2}$ and
$\overline{\cdot}^{\,t\,*}= \overline{\cdot}^{\,t+\Delta/2}$, respectively.

The Leap-frog time stepping given by \autoref{eq:DOM_nxt} can be defined as:
\[
  % \label{eq:ALGOS_LF}
  \frac{\partial q}{\partial t}
  \equiv \frac{1}{\rdt} \overline{ \delta_{t+\rdt/2}[q]}^{\,t}
  =         \frac{q^{t+\rdt}-q^{t-\rdt}}{2\rdt}
\]
Note that \autoref{chap:LF} shows that the leapfrog time step is $\rdt$,
not $2\rdt$ as it can be found sometimes in literature.
The leap-Frog time stepping is a second order centered scheme.
As such it respects the quadratic invariant in integral forms, \ie\ the following continuous property,
\[
  % \label{eq:ALGOS_Energy}
  \int_{t_0}^{t_1} {q\, \frac{\partial q}{\partial t} \;dt}
  =\int_{t_0}^{t_1} {\frac{1}{2}\, \frac{\partial q^2}{\partial t} \;dt}
  =  \frac{1}{2} \left( {q_{t_1}}^2 - {q_{t_0}}^2 \right) ,
\]
is satisfied in discrete form.
Indeed,
\[
  \begin{split}
    \int_{t_0}^{t_1} {q\, \frac{\partial q}{\partial t} \;dt}
    &\equiv \sum\limits_{0}^{N}
    {\frac{1}{\rdt} q^t \ \overline{ \delta_{t+\rdt/2}[q]}^{\,t} \ \rdt}
    \equiv \sum\limits_{0}^{N}  { q^t \ \overline{ \delta_{t+\rdt/2}[q]}^{\,t} } \\
    &\equiv \sum\limits_{0}^{N}  { \overline{q}^{\,t+\Delta/2}{ \delta_{t+\rdt/2}[q]}}
    \equiv \sum\limits_{0}^{N}  { \frac{1}{2} \delta_{t+\rdt/2}[q^2] }\\
    &\equiv \sum\limits_{0}^{N}  { \frac{1}{2} \delta_{t+\rdt/2}[q^2] }
    \equiv \frac{1}{2} \left( {q_{t_1}}^2 - {q_{t_0}}^2 \right)
  \end{split}
\]
NB here pb of boundary condition when applying the adjoint!
In space, setting to 0 the quantity in land area is sufficient to get rid of the boundary condition
(equivalently of the boundary value of the integration by part).
In time this boundary condition is not physical and \textbf{add something here!!!}

%% =================================================================================================
\section{Lateral diffusion operator}

%% =================================================================================================
\subsection{Griffies iso-neutral diffusion operator}

Let try to define a scheme that get its inspiration from the \citet{griffies.gnanadesikan.ea_JPO98} scheme,
but is formulated within the \NEMO\ framework
(\ie\ using scale factors rather than grid-size and having a position of $T$-points that
is not necessary in the middle of vertical velocity points, see \autoref{fig:DOM_zgr_e3}).

In the formulation \autoref{eq:TRA_ldf_iso} introduced in 1995 in OPA, the ancestor of \NEMO,
the off-diagonal terms of the small angle diffusion tensor contain several double spatial averages of a gradient,
for example $\overline{\overline{\delta_k \cdot}}^{\,i,k}$.
It is apparent that the combination of a $k$ average and a $k$ derivative of the tracer allows for
the presence of grid point oscillation structures that will be invisible to the operator.
These structures are \textit{computational modes}.
They will not be damped by the iso-neutral operator, and even possibly amplified by it.
In other word, the operator applied to a tracer does not warranties the decrease of its global average variance.
To circumvent this, we have introduced a smoothing of the slopes of the iso-neutral surfaces
(see \autoref{chap:LDF}).
Nevertheless, this technique works fine for $T$ and $S$ as they are active tracers
(\ie\ they enter the computation of density), but it does not work for a passive tracer.
\citep{griffies.gnanadesikan.ea_JPO98} introduce a different way to discretise the off-diagonal terms that
nicely solve the problem.
The idea is to get rid of combinations of an averaged in one direction combined with
a derivative in the same direction by considering triads.
For example in the (\textbf{i},\textbf{k}) plane, the four triads are defined at the $(i,k)$ $T$-point as follows:
\begin{equation}
  \label{eq:ALGOS_Gf_triads}
  _i^k \mathbb{T}_{i_p}^{k_p} (T)
  = \frac{1}{4} \ {b_u}_{\,i+i_p}^{\,k}  \  A_i^k  	\left(
    \frac{ \delta_{i + i_p}[T^k] }{ {e_{1u}}_{\,i + i_p}^{\,k} }
    -\ {_i^k \mathbb{R}_{i_p}^{k_p}} \ \frac{ \delta_{k+k_p} [T^i] }{ {e_{3w}}_{\,i}^{\,k+k_p} }
  \right)
\end{equation}
where the indices $i_p$ and $k_p$ define the four triads and take the following value:
$i_p = -1/2$ or $1/2$ and $k_p = -1/2$ or $1/2$,
$b_u= e_{1u}\,e_{2u}\,e_{3u}$ is the volume of $u$-cells,
$A_i^k$ is the lateral eddy diffusivity coefficient defined at $T$-point,
and $_i^k \mathbb{R}_{i_p}^{k_p}$ is the slope associated with each triad:
\begin{equation}
  \label{eq:ALGOS_Gf_slopes}
  _i^k \mathbb{R}_{i_p}^{k_p}
  =\frac{ {e_{3w}}_{\,i}^{\,k+k_p}} { {e_{1u}}_{\,i+i_p}^{\,k}} \ \frac
  {\left(\alpha / \beta \right)_i^k  \ \delta_{i + i_p}[T^k] - \delta_{i + i_p}[S^k] }
  {\left(\alpha / \beta \right)_i^k  \ \delta_{k+k_p}[T^i ] - \delta_{k+k_p}[S^i ] }
\end{equation}
Note that in \autoref{eq:ALGOS_Gf_slopes} we use the ratio $\alpha / \beta$ instead of
multiplying the temperature derivative by $\alpha$ and the salinity derivative by $\beta$.
This is more efficient as the ratio $\alpha / \beta$ can to be evaluated directly.

Note that in \autoref{eq:ALGOS_Gf_triads}, we chose to use ${b_u}_{\,i+i_p}^{\,k}$ instead of ${b_{uw}}_{\,i+i_p}^{\,k+k_p}$.
This choice has been motivated by the decrease of tracer variance and
the presence of partial cell at the ocean bottom (see \autoref{subsec:ALGOS_Gf_operator}).

\begin{figure}[!ht]
  \centering
  %\includegraphics[width=0.66\textwidth]{ALGOS_ISO_triad}
  \caption[Triads used in the Griffies's like iso-neutral diffision scheme for
    $u$- and $w$-components)]{
    Triads used in the Griffies's like iso-neutral diffision scheme for
    $u$-component (upper panel) and $w$-component (lower panel).}
  \label{fig:ALGOS_ISO_triad}
\end{figure}

The four iso-neutral fluxes associated with the triads are defined at $T$-point.
They take the following expression:
\begin{flalign*}
  % \label{eq:ALGOS_Gf_fluxes}
  \begin{split}
    {_i^k {\mathbb{F}_u}_{i_p}^{k_p} } (T)
    &= \ \; \qquad  \quad    { _i^k \mathbb{T}_{i_p}^{k_p} }(T) \;\ / \ { {e_{1u}}_{\,i+i_p}^{\,k}}    \\
    {_i^k {\mathbb{F}_w}_{i_p}^{k_p} } (T)
    &=  -\; { _i^k \mathbb{R}_{i_p}^{k_p} }
    \ \; { _i^k \mathbb{T}_{i_p}^{k_p} }(T) \;\ / \ { {e_{3w}}_{\,i}^{\,k+k_p}}
  \end{split}
\end{flalign*}

The resulting iso-neutral fluxes at $u$- and $w$-points are then given by
the sum of the fluxes that cross the $u$- and $w$-face (\autoref{fig:TRIADS_ISO_triad}):
\begin{flalign}
  \label{eq:ALGOS_iso_flux}
  \textbf{F}_{iso}(T)
  &\equiv  \sum_{\substack{i_p,\,k_p}}
  \begin{pmatrix}
    {_{i+1/2-i_p}^k {\mathbb{F}_u}_{i_p}^{k_p} } (T) \\ \\
    {_i^{k+1/2-k_p} {\mathbb{F}_w}_{i_p}^{k_p} } (T)
  \end{pmatrix}
  \notag \\
  &  \notag \\
  &\equiv  \sum_{\substack{i_p,\,k_p}}
  \begin{pmatrix}
    && { _{i+1/2-i_p}^k \mathbb{T}_{i_p}^{k_p} }(T) \;\ / \ { {e_{1u}}_{\,i+1/2}^{\,k} } \\ \\
    & -\; { _i^{k+1/2-k_p} \mathbb{R}_{i_p}^{k_p} }
    & {_i^{k+1/2-k_p} \mathbb{T}_{i_p}^{k_p} }(T) \;\ / \ { {e_{3w}}_{\,i}^{\,k+1/2} }
  \end{pmatrix}      % \\
  % &\\
  % &\equiv  \sum_{\substack{i_p,\,k_p}}
  % \begin{pmatrix}
  %   \qquad  \qquad  \qquad
  %   \frac{1}{ {e_{1u}}_{\,i+1/2}^{\,k} }  \ \;
  %   { _{i+1/2-i_p}^k \mathbb{T}_{i_p}^{k_p} }(T)\\
  %   \\
  %   -\frac{1}{ {e_{3w}}_{\,i}^{\,k+1/2} }  \ \;
  %   { _i^{k+1/2-k_p} \mathbb{R}_{i_p}^{k_p} } \ \;
  %   {_i^{k+1/2-k_p} \mathbb{T}_{i_p}^{k_p} }(T)\\
  % \end{pmatrix}
\end{flalign}
resulting in a iso-neutral diffusion tendency on temperature given by
the divergence of the sum of all the four triad fluxes:
\begin{equation}
  \label{eq:ALGOS_Gf_operator}
  D_l^T = \frac{1}{b_T}  \sum_{\substack{i_p,\,k_p}} \left\{
    \delta_{i} \left[{_{i+1/2-i_p}^k {\mathbb{F}_u }_{i_p}^{k_p}} \right]
    + \delta_{k} \left[ {_i^{k+1/2-k_p} {\mathbb{F}_w}_{i_p}^{k_p}} \right]   \right\}
\end{equation}
where $b_T= e_{1T}\,e_{2T}\,e_{3T}$ is the volume of $T$-cells.

This expression of the iso-neutral diffusion has been chosen in order to satisfy the following six properties:
\begin{description}
\item [Horizontal diffusion] The discretization of the diffusion operator recovers the traditional five-point Laplacian in the limit of flat iso-neutral direction:
  \[
    % \label{eq:ALGOS_Gf_property1a}
    D_l^T = \frac{1}{b_T}  \ \delta_{i}
    \left[ \frac{e_{2u}\,e_{3u}}{e_{1u}} \; \overline{A}^{\,i} \; \delta_{i+1/2}[T] \right]
    \qquad  \text{when} \quad
    { _i^k \mathbb{R}_{i_p}^{k_p} }=0
  \]
\item [Implicit treatment in the vertical] In the diagonal term associated with the vertical divergence of the iso-neutral fluxes
  \ie\ the term associated with a second order vertical derivative)
  appears only tracer values associated with a single water column.
  This is of paramount importance since it means that
  the implicit in time algorithm for solving the vertical diffusion equation can be used to evaluate this term.
  It is a necessity since the vertical eddy diffusivity associated with this term,
  \[
	 \sum_{\substack{i_p, \,k_p}} \left\{
		A_i^k \; \left(_i^k \mathbb{R}_{i_p}^{k_p}\right)^2
    \right\}
  \]
  can be quite large.
\item [Pure iso-neutral operator] The iso-neutral flux of locally referenced potential density is zero, \ie
  \begin{align*}
    % \label{eq:ALGOS_Gf_property2}
    \begin{matrix}
      &{_i^k {\mathbb{F}_u}_{i_p}^{k_p} (\rho)}
      &=    &\alpha_i^k   &{_i^k {\mathbb{F}_u}_{i_p}^{k_p} } (T)
      &- \ \;  \beta _i^k    &{_i^k {\mathbb{F}_u}_{i_p}^{k_p} } (S) & = \ 0   \\
      &{_i^k {\mathbb{F}_w}_{i_p}^{k_p} (\rho)}
      &=    &\alpha_i^k   &{_i^k {\mathbb{F}_w}_{i_p}^{k_p} } (T)
      &- \  \; \beta _i^k    &{_i^k {\mathbb{F}_w}_{i_p}^{k_p} } (S)  &= \ 0
    \end{matrix}
  \end{align*}
  This result is trivially obtained using the \autoref{eq:ALGOS_Gf_triads} applied to $T$ and $S$ and
  the definition of the triads' slopes \autoref{eq:ALGOS_Gf_slopes}.
\item [Conservation of tracer] The iso-neutral diffusion term conserve the total tracer content, \ie
  \[
    % \label{eq:ALGOS_Gf_property1}
    \sum_{i,j,k} \left\{ D_l^T \ b_T \right\} = 0
  \]
This property is trivially satisfied since the iso-neutral diffusive operator is written in flux form.
\item [Decrease of tracer variance] The iso-neutral diffusion term does not increase the total tracer variance, \ie
  \[
    % \label{eq:ALGOS_Gf_property1}
    \sum_{i,j,k} \left\{ T \ D_l^T \ b_T \right\} \leq 0
  \]
The property is demonstrated in the \autoref{subsec:ALGOS_Gf_operator}.
It is a key property for a diffusion term.
It means that the operator is also a dissipation term,
\ie\ it is a sink term for the square of the quantity on which it is applied.
It therfore ensures that, when the diffusivity coefficient is large enough,
the field on which it is applied become free of grid-point noise.
\item [Self-adjoint operator] The iso-neutral diffusion operator is self-adjoint, \ie
  \[
    % \label{eq:ALGOS_Gf_property1}
    \sum_{i,j,k} \left\{ S \ D_l^T \ b_T \right\} = \sum_{i,j,k} \left\{ D_l^S \ T \ b_T \right\}
  \]
In other word, there is no needs to develop a specific routine from the adjoint of this operator.
We just have to apply the same routine.
This properties can be demonstrated quite easily in a similar way the "non increase of tracer variance" property
has been proved (see \autoref{apdx:ALGOS_Gf_operator}).
\end{description}

%% =================================================================================================
\subsection{Eddy induced velocity and skew flux formulation}

When Gent and McWilliams [1990] diffusion is used,
an additional advection term is added.
The associated velocity is the so called eddy induced velocity,
the formulation of which depends on the slopes of iso-neutral surfaces.
Contrary to the case of iso-neutral mixing, the slopes used here are referenced to the geopotential surfaces,
\ie\ \autoref{eq:LDF_slp_geo} is used in $z$-coordinate,
and the sum \autoref{eq:LDF_slp_geo} + \autoref{eq:LDF_slp_iso} in $z^*$ or $s$-coordinates.

The eddy induced velocity is given by:
\begin{equation}
  \label{eq:ALGOS_eiv_v}
  \begin{split}
    u^* & = - \frac{1}{e_2\,e_{3}}          \;\partial_k \left( e_2 \, A_e \; r_i  \right)
    = - \frac{1}{e_3}                     \;\partial_k \left(           A_e \; r_i  \right)            \\
    v^* & = - \frac{1}{e_1\,e_3}\;             \partial_k \left( e_1 \, A_e \; r_j  \right)
    = - \frac{1}{e_3}                     \;\partial_k \left(           A_e \; r_j  \right)             \\
    w^* & =    \frac{1}{e_1\,e_2}\; \left\{   \partial_i  \left( e_2 \, A_e \; r_i  \right)
      + \partial_j  \left( e_1 \, A_e \;r_j   \right) \right\}
  \end{split}
\end{equation}
where $A_{e}$ is the eddy induced velocity coefficient,
and $r_i$ and $r_j$ the slopes between the iso-neutral and the geopotential surfaces.
\cmtgm{Wrong: to be modified with 2 2D streamfunctions}
In other words, the eddy induced velocity can be derived from a vector streamfuntion, $\phi$,
which is given by $\phi = A_e\,\textbf{r}$ as $\textbf{U}^*  = \textbf{k} \times \nabla \phi$.

A traditional way to implement this additional advection is to add it to the eulerian velocity prior to
compute the tracer advection.
This allows us to take advantage of all the advection schemes offered for the tracers
(see \autoref{sec:TRA_adv}) and not just a $2^{nd}$ order advection scheme.
This is particularly useful for passive tracers where
\emph{positivity} of the advection scheme is of paramount importance.
% give here the expression using the triads. It is different from the one given in \autoref{eq:LDF_eiv}
% see just below a copy of this equation:
%\begin{equation} \label{eq:ALGOS_ldfeiv}
%\begin{split}
% u^* & = \frac{1}{e_{2u}e_{3u}}\; \delta_k \left[e_{2u} \, A_{uw}^{eiv} \; \overline{r_{1w}}^{\,i+1/2} \right]\\
% v^* & = \frac{1}{e_{1u}e_{3v}}\; \delta_k \left[e_{1v} \, A_{vw}^{eiv} \; \overline{r_{2w}}^{\,j+1/2} \right]\\
%w^* & = \frac{1}{e_{1w}e_{2w}}\; \left\{ \delta_i \left[e_{2u} \, A_{uw}^{eiv} \; \overline{r_{1w}}^{\,i+1/2} \right] + %\delta_j \left[e_{1v} \, A_{vw}^{eiv} \; \overline{r_{2w}}^{\,j+1/2} \right] \right\} \\
%\end{split}
%\end{equation}
\[
  % \label{eq:ALGOS_eiv_vd}
  \textbf{F}_{eiv}^T   \equiv   \left(
    \begin{aligned}
      \sum_{\substack{i_p,\,k_p}} &
      +{e_{2u}}_{i+1/2-i_p}^{k}                                  \ \ {A_{e}}_{i+1/2-i_p}^{k}
      \ \ \ { _{i+1/2-i_p}^k \mathbb{R}_{i_p}^{k_p} }    \ \ \delta_{k+k_p}[T_{i+1/2-i_p}] \\ \\
      \sum_{\substack{i_p,\,k_p}} &
      - {e_{2u}}_i^{k+1/2-k_p}                                      \ {A_{e}}_i^{k+1/2-k_p}
      \ \ { _i^{k+1/2-k_p} \mathbb{R}_{i_p}^{k_p} }    \ \delta_{i+i_p}[T^{k+1/2-k_p}]
    \end{aligned}
  \right)
\]

\citep{griffies_JPO98} introduces another way to implement the eddy induced advection, the so-called skew form.
It is based on a transformation of the advective fluxes using the non-divergent nature of the eddy induced velocity.
For example in the (\textbf{i},\textbf{k}) plane, the tracer advective fluxes can be transformed as follows:
\begin{flalign*}
  \begin{split}
    \textbf{F}_{eiv}^T =
    \begin{pmatrix}
      {e_{2}\,e_{3}\;  u^*} 	 	\\
 		{e_{1}\,e_{2}\; w^*}
    \end{pmatrix}
    \;   T
    &=
    \begin{pmatrix}
      { - \partial_k \left( e_{2} \, A_{e} \; r_i \right) \; T \;} 	 	\\
 		{+ \partial_i  \left( e_{2} \, A_{e} \; r_i \right) \; T \;}
    \end{pmatrix}
    \\
    &=
    \begin{pmatrix}
      { - \partial_k \left( e_{2} \, A_{e} \; r_i  \; T \right) \;}  \\
 		{+ \partial_i  \left( e_{2} \, A_{e} \; r_i  \; T \right) \;}
    \end{pmatrix}
    +
    \begin{pmatrix}
      {+ e_{2} \, A_{e} \; r_i  \; \partial_k T}  \\
 		{ - e_{2} \, A_{e} \; r_i  \; \partial_i  T}
    \end{pmatrix}
  \end{split}
\end{flalign*}
and since the eddy induces velocity field is no-divergent,
we end up with the skew form of the eddy induced advective fluxes:
\begin{equation}
  \label{eq:ALGOS_eiv_skew_continuous}
  \textbf{F}_{eiv}^T =
  \begin{pmatrix}
    {+ e_{2} \, A_{e} \; r_i  \; \partial_k T}   \\
    { - e_{2} \, A_{e} \; r_i  \; \partial_i  T}
  \end{pmatrix}
\end{equation}
The tendency associated with eddy induced velocity is then simply the divergence of
the \autoref{eq:ALGOS_eiv_skew_continuous} fluxes.
It naturally conserves the tracer content, as it is expressed in flux form and,
as the advective form, it preserves the tracer variance.
Another interesting property of \autoref{eq:ALGOS_eiv_skew_continuous} form is that when $A=A_e$,
a simplification occurs in the sum of the iso-neutral diffusion and eddy induced velocity terms:
\begin{flalign*}
  % \label{eq:ALGOS_eiv_skew+eiv_continuous}
  \textbf{F}_{iso}^T + \textbf{F}_{eiv}^T &=
  \begin{pmatrix}
    + \frac{e_2\,e_3\,}{e_1} A \;\partial_i T -  e_2 \, A \; r_i                              \;\partial_k T   \\
    -  e_2 \, A_{e} \; r_i           \;\partial_i T + \frac{e_1\,e_2}{e_3} \, A \; r_i^2 \;\partial_k T
  \end{pmatrix}
  +
  \begin{pmatrix}
    {+ e_{2} \, A_{e} \; r_i  \; \partial_k T}   \\
    { - e_{2} \, A_{e} \; r_i  \; \partial_i  T}
  \end{pmatrix}
  \\
  &=
  \begin{pmatrix}
    + \frac{e_2\,e_3\,}{e_1} A \;\partial_i T    \\
    -  2\; e_2 \, A_{e} \; r_i      \;\partial_i T + \frac{e_1\,e_2}{e_3} \, A \; r_i^2 \;\partial_k T
  \end{pmatrix}
\end{flalign*}
The horizontal component reduces to the one use for an horizontal laplacian operator and
the vertical one keeps the same complexity, but not more.
This property has been used to reduce the computational time \citep{griffies_JPO98},
but it is not of practical use as usually $A \neq A_e$.
Nevertheless this property can be used to choose a discret form of \autoref{eq:ALGOS_eiv_skew_continuous} which
is consistent with the iso-neutral operator \autoref{eq:ALGOS_Gf_operator}.
Using the slopes \autoref{eq:ALGOS_Gf_slopes} and defining $A_e$ at $T$-point(\ie\ as $A$,
the eddy diffusivity coefficient), the resulting discret form is given by:
\begin{equation}
  \label{eq:ALGOS_eiv_skew}
  \textbf{F}_{eiv}^T   \equiv   \frac{1}{4} \left(
    \begin{aligned}
      \sum_{\substack{i_p,\,k_p}} &
      +{e_{2u}}_{i+1/2-i_p}^{k}                                  \ \ {A_{e}}_{i+1/2-i_p}^{k}
      \ \ \ { _{i+1/2-i_p}^k \mathbb{R}_{i_p}^{k_p} }    \ \ \delta_{k+k_p}[T_{i+1/2-i_p}] \\ \\
      \sum_{\substack{i_p,\,k_p}} &
      - {e_{2u}}_i^{k+1/2-k_p}                                      \ {A_{e}}_i^{k+1/2-k_p}
      \ \ { _i^{k+1/2-k_p} \mathbb{R}_{i_p}^{k_p} }    \ \delta_{i+i_p}[T^{k+1/2-k_p}]
    \end{aligned}
  \right)
\end{equation}
Note that \autoref{eq:ALGOS_eiv_skew} is valid in $z$-coordinate with or without partial cells.
In $z^*$ or $s$-coordinate, the slope between the level and the geopotential surfaces must be added to
$\mathbb{R}$ for the discret form to be exact.

Such a choice of discretisation is consistent with the iso-neutral operator as
it uses the same definition for the slopes.
It also ensures the conservation of the tracer variance (see \autoref{subsec:ALGOS_eiv_skew}),
\ie\ it does not include a diffusive component but is a "pure" advection term.

%% =================================================================================================
\subsection{Discrete invariants of the iso-neutral diffrusion}
\label{subsec:ALGOS_Gf_operator}

Demonstration of the decrease of the tracer variance in the (\textbf{i},\textbf{j}) plane.

This part will be moved in an Appendix.

The continuous property to be demonstrated is:
\[
  \int_D  D_l^T \; T \;dv   \leq 0
\]
The discrete form of its left hand side is obtained using \autoref{eq:TRIADS_iso_flux}

\begin{align*}
  &\int_D  D_l^T \; T \;dv \equiv  \sum_{i,k} \left\{ T \ D_l^T \ b_T \right\}    \\
  &\equiv + \sum_{i,k} \sum_{\substack{i_p,\,k_p}} \left\{
    \delta_{i} \left[{_{i+1/2-i_p}^k {\mathbb{F}_u }_{i_p}^{k_p}} \right]
    + \delta_{k} \left[ {_i^{k+1/2-k_p} {\mathbb{F}_w}_{i_p}^{k_p}} \right]  \ T \right\}    \\
  &\equiv  - \sum_{i,k} \sum_{\substack{i_p,\,k_p}} \left\{
    {_{i+1/2-i_p}^k {\mathbb{F}_u }_{i_p}^{k_p}} \ \delta_{i+1/2} [T]
    + {_i^{k+1/2-k_p} {\mathbb{F}_w}_{i_p}^{k_p}}  \ \delta_{k+1/2} [T]   \right\}      \\
  &\equiv -\sum_{i,k} \sum_{\substack{i_p,\,k_p}} \left\{
    \frac{ _{i+1/2-i_p}^k \mathbb{T}_{i_p}^{k_p} (T) }{ {e_{1u}}_{\,i+1/2}^{\,k} }  \ \delta_{i+1/2} [T]
    - { _i^{k+1/2-k_p} \mathbb{R}_{i_p}^{k_p} } \ \;
    \frac{ _i^{k+1/2-k_p} \mathbb{T}_{i_p}^{k_p} (T) }{ {e_{3w}}_{\,i}^{\,k+1/2}  } \ \delta_{k+1/2} [T]
    \right\}      \\
    %
  \allowdisplaybreaks
  \intertext{ Expending the summation on $i_p$ and $k_p$, it becomes:}
  %
  &\equiv -\sum_{i,k}
    \begin{Bmatrix}
      &\ \ \Bigl(  { _{i+1}^{k} \mathbb{T}_{-1/2}^{-1/2} (T) }
      &\frac{ \delta_{i +1/2} [T] }{{e_{1u} }_{\,i+1/2}^{\,k}}
      & -\ \ {_{i}^{k+1} \mathbb{R}_{-1/2}^{-1/2}}
      &      {_{i}^{k+1} \mathbb{T}_{-1/2}^{-1/2} (T) }
      &\frac{ \delta_{k+1/2} [T] }{{e_{3w}}_{\,i}^{\,k+1/2}}     \Bigr)
      & \\
      &+\Bigl( \ \;\; { _i^k \mathbb{T}_{+1/2}^{-1/2} (T) }
      &\frac{ \delta_{i +1/2} [T] }{{e_{1u} }_{\,i+1/2}^{\,k}}
      & -\ \ {_i^{k+1} \mathbb{R}_{+1/2}^{-1/2}}
      & { _i^{k+1} \mathbb{T}_{+1/2}^{-1/2} (T) }
      &\frac{ \delta_{k+1/2} [T] }{{e_{3w}}_{\,i}^{\,k+1/2}}      \Bigr)
      & \\
      &+\Bigl(  { _{i+1}^{k} \mathbb{T}_{-1/2}^{+1/2} (T) }
      &\frac{ \delta_{i +1/2} [T] }{{e_{1u} }_{\,i+1/2}^{\,k}}
      & -\ \ \ \;\;{_{i}^{k} \mathbb{R}_{-1/2}^{+1/2}}
      &      \ \;\;{_{i}^{k} \mathbb{T}_{-1/2}^{+1/2} (T) }
      &\frac{ \delta_{k+1/2} [T] }{{e_{3w}}_{\,i}^{\,k+1/2}}     \Bigr)
      & \\
      &+\Bigl( \ \;\; { _{i}^{k} \mathbb{T}_{+1/2}^{+1/2} (T) }
      &\frac{ \delta_{i +1/2} [T] }{{e_{1u} }_{\,i+1/2}^{\,k}}
      & -\ \ \ \;\;{_{i}^{k} \mathbb{R}_{+1/2}^{+1/2}}
      &      \ \;\;{_{i}^{k} \mathbb{T}_{+1/2}^{+1/2} (T) }
      &\frac{ \delta_{k+1/2} [T] }{{e_{3w}}_{\,i}^{\,k+1/2}}     \Bigr)   \\
    \end{Bmatrix}
    %
  \allowdisplaybreaks
  \intertext{
  The summation is done over all $i$ and $k$ indices,
  it is therefore possible to introduce a shift of $-1$ either in $i$ or $k$ direction in order to
  regroup all the terms of the summation by triad at a ($i$,$k$) point.
  In other words, we regroup all the terms in the neighbourhood that contain a triad at the same ($i$,$k$) indices.
  It becomes:
  }
  %
  &\equiv -\sum_{i,k}
    \begin{Bmatrix}
      &\ \ \Bigl(  {_i^k \mathbb{T}_{-1/2}^{-1/2} (T) }
      &\frac{ \delta_{i -1/2} [T] }{{e_{1u} }_{\,i-1/2}^{\,k}}
      & -\ \ {_i^k \mathbb{R}_{-1/2}^{-1/2}}
      &      {_i^k \mathbb{T}_{-1/2}^{-1/2} (T) }
      &\frac{ \delta_{k-1/2} [T] }{{e_{3w}}_{\,i}^{\,k-1/2}}     \Bigr)
      & \\
      &+\Bigl(  { _i^k \mathbb{T}_{+1/2}^{-1/2} (T) }
      &\frac{ \delta_{i +1/2} [T] }{{e_{1u} }_{\,i+1/2}^{\,k}}
      & -\ \ {_i^k \mathbb{R}_{+1/2}^{-1/2}}
      &      { _i^k \mathbb{T}_{+1/2}^{-1/2} (T) }
      &\frac{ \delta_{k-1/2} [T] }{{e_{3w}}_{\,i}^{\,k-1/2}}      \Bigr)
      & \\
      &+\Bigl(  {_i^k \mathbb{T}_{-1/2}^{+1/2} (T) }
      &\frac{ \delta_{i -1/2} [T] }{{e_{1u} }_{\,i-1/2}^{\,k}}
      & -\ \ {_i^k \mathbb{R}_{-1/2}^{+1/2}}
      &      {_i^k \mathbb{T}_{-1/2}^{+1/2} (T) }
      &\frac{ \delta_{k+1/2} [T] }{{e_{3w}}_{\,i}^{\,k+1/2}}     \Bigr)
      & \\
      &+\Bigl( { _i^k \mathbb{T}_{+1/2}^{+1/2} (T) }
      &\frac{ \delta_{i +1/2} [T] }{{e_{1u} }_{\,i+1/2}^{\,k}}
      & -\ \ {_i^k \mathbb{R}_{+1/2}^{+1/2}}
      &      {_i^k \mathbb{T}_{+1/2}^{+1/2} (T) }
      &\frac{ \delta_{k+1/2} [T] }{{e_{3w}}_{\,i}^{\,k+1/2}}     \Bigr)   \\
    \end{Bmatrix}   \\
    %
  \allowdisplaybreaks
  \intertext{
  Then outing in factor the triad in each of the four terms of the summation and
  substituting the triads by their expression given in \autoref{eq:ALGOS_Gf_triads}.
  It becomes:
  }
  %
  &\equiv -\sum_{i,k}
    \begin{Bmatrix}
      &\ \ \Bigl(  \frac{ \delta_{i -1/2} [T] }{{e_{1u} }_{\,i-1/2}^{\,k}}
      & -\ \ {_i^k \mathbb{R}_{-1/2}^{-1/2}}
      &\frac{ \delta_{k-1/2} [T] }{{e_{3w}}_{\,i}^{\,k-1/2}}     \Bigr)^2
      & \frac{1}{4} \ {b_u}_{\,i-1/2}^{\,k}  \  A_i^k
      & \\
      &+\Bigl(  \frac{ \delta_{i +1/2} [T] }{{e_{1u} }_{\,i+1/2}^{\,k}}
      & -\ \ {_i^k \mathbb{R}_{+1/2}^{-1/2}}
      &\frac{ \delta_{k-1/2} [T] }{{e_{3w}}_{\,i}^{\,k-1/2}}      \Bigr)^2
      & \frac{1}{4} \ {b_u}_{\,i+1/2}^{\,k}  \  A_i^k
      & \\
      &+\Bigl(  \frac{ \delta_{i -1/2} [T] }{{e_{1u} }_{\,i-1/2}^{\,k}}
      & -\ \ {_i^k \mathbb{R}_{-1/2}^{+1/2}}
      &\frac{ \delta_{k+1/2} [T] }{{e_{3w}}_{\,i}^{\,k+1/2}}     \Bigr)^2
      & \frac{1}{4} \ {b_u}_{\,i-1/2}^{\,k}  \  A_i^k
      & \\
      &+\Bigl( \frac{ \delta_{i +1/2} [T] }{{e_{1u} }_{\,i+1/2}^{\,k}}
      & -\ \ {_i^k \mathbb{R}_{+1/2}^{+1/2}}
      &\frac{ \delta_{k+1/2} [T] }{{e_{3w}}_{\,i}^{\,k+1/2}}     \Bigr)^2
      & \frac{1}{4} \ {b_u}_{\,i+1/2}^{\,k}  \  A_i^k      \\
    \end{Bmatrix}
  \\
  & \\
  %
  &\equiv - \sum_{i,k} \sum_{\substack{i_p,\,k_p}} \left\{
    \begin{matrix}
      &\Bigl( \frac{ \delta_{i +i_p} [T] }{{e_{1u} }_{\,i+i_p}^{\,k}}
      & -\ \ {_i^k \mathbb{R}_{i_p}^{k_p}}
      &\frac{ \delta_{k+k_p} [T] }{{e_{3w}}_{\,i}^{\,k+k_p}}     \Bigr)^2
      & \frac{1}{4} \ {b_u}_{\,i+i_p}^{\,k}  \  A_i^k   \ \
    \end{matrix}
        \right\}
        \quad   \leq 0
\end{align*}
The last inequality is obviously obtained as we succeed in obtaining a negative summation of square quantities.

Note that, if instead of multiplying $D_l^T$ by $T$, we were using another tracer field, let say $S$,
then the previous demonstration would have let to:
\begin{align*}
  \int_D  S \; D_l^T  \;dv &\equiv  \sum_{i,k} \left\{ S \ D_l^T \ b_T \right\}    \\
                           &\equiv - \sum_{i,k} \sum_{\substack{i_p,\,k_p}} \left\{
                             \left( \frac{ \delta_{i +i_p} [S] }{{e_{1u} }_{\,i+i_p}^{\,k}}
                             - {_i^k \mathbb{R}_{i_p}^{k_p}}
                             \frac{ \delta_{k+k_p} [S] }{{e_{3w}}_{\,i}^{\,k+k_p}}     \right)  \right. \\
                           & \qquad \qquad \qquad \ \left.
                             \left( \frac{ \delta_{i +i_p} [T] }{{e_{1u} }_{\,i+i_p}^{\,k}}
                             - {_i^k \mathbb{R}_{i_p}^{k_p}}
                             \frac{ \delta_{k+k_p} [T] }{{e_{3w}}_{\,i}^{\,k+k_p}}     \right)
                             \frac{1}{4} \ {b_u}_{\,i+i_p}^{\,k}  \  A_i^k   \
                             \right\}
                             %
                             \allowdisplaybreaks
                             \intertext{
                             which, by applying the same operation as before but in reverse order, leads to:
                             }
                             %
                           &\equiv  \sum_{i,k} \left\{ D_l^S \ T \ b_T \right\}
\end{align*}
This means that the iso-neutral operator is self-adjoint.
There is no need to develop a specific to obtain it.

%% =================================================================================================
\subsection{Discrete invariants of the skew flux formulation}
\label{subsec:ALGOS_eiv_skew}

Demonstration for the conservation of the tracer variance in the (\textbf{i},\textbf{j}) plane.

This have to be moved in an Appendix.

The continuous property to be demonstrated is:
\begin{align*}
  \int_D \nabla \cdot \textbf{F}_{eiv}(T) \; T \;dv  \equiv 0
\end{align*}
The discrete form of its left hand side is obtained using \autoref{eq:ALGOS_eiv_skew}
\begin{align*}
  \sum\limits_{i,k} \sum_{\substack{i_p,\,k_p}}  \Biggl\{   \;\;
  \delta_i  &\left[
              {e_{2u}}_{i+i_p+1/2}^{k}                                  \;\ \ {A_{e}}_{i+i_p+1/2}^{k}
              \ \ \ { _{i+i_p+1/2}^k \mathbb{R}_{-i_p}^{k_p} }   \quad \delta_{k+k_p}[T_{i+i_p+1/2}]
              \right] \; T_i^k      \\
  - \delta_k &\left[
               {e_{2u}}_i^{k+k_p+1/2}                                     \ \ {A_{e}}_i^{k+k_p+1/2}
               \ \ { _i^{k+k_p+1/2} \mathbb{R}_{i_p}^{-k_p} }   \ \ \delta_{i+i_p}[T^{k+k_p+1/2}]
               \right] \; T_i^k      \         \Biggr\}
\end{align*}
apply the adjoint of delta operator, it becomes
\begin{align*}
  \sum\limits_{i,k} \sum_{\substack{i_p,\,k_p}}  \Biggl\{   \;\;
  &\left(
    {e_{2u}}_{i+i_p+1/2}^{k}                                  \;\ \ {A_{e}}_{i+i_p+1/2}^{k}
    \ \ \ { _{i+i_p+1/2}^k \mathbb{R}_{-i_p}^{k_p} }   \quad \delta_{k+k_p}[T_{i+i_p+1/2}]
    \right) \; \delta_{i+1/2}[T^{k}]      \\
  - &\left(
      {e_{2u}}_i^{k+k_p+1/2}                                     \ \ {A_{e}}_i^{k+k_p+1/2}
      \ \ { _i^{k+k_p+1/2} \mathbb{R}_{i_p}^{-k_p} }   \ \ \delta_{i+i_p}[T^{k+k_p+1/2}]
      \right) \; \delta_{k+1/2}[T_{i}]       \         \Biggr\}
\end{align*}
Expending the summation on $i_p$ and $k_p$, it becomes:
\begin{align*}
  \begin{matrix}
    &\sum\limits_{i,k}   \Bigl\{
    &+{e_{2u}}_{i+1}^{k}                             &{A_{e}}_{i+1    }^{k}
    &\ {_{i+1}^k \mathbb{R}_{- 1/2}^{-1/2}} &\delta_{k-1/2}[T_{i+1}]    &\delta_{i+1/2}[T^{k}]   &\\
    &&+{e_{2u}}_i^{k\ \ \ \:}                            &{A_{e}}_{i}^{k\ \ \ \:}
    &\ {\ \ \;_i^k \mathbb{R}_{+1/2}^{-1/2}}  &\delta_{k-1/2}[T_{i\ \ \ \;}]  &\delta_{i+1/2}[T^{k}] &\\
    &&+{e_{2u}}_{i+1}^{k}                             &{A_{e}}_{i+1    }^{k}
    &\ {_{i+1}^k \mathbb{R}_{- 1/2}^{+1/2}} &\delta_{k+1/2}[T_{i+1}]     &\delta_{i+1/2}[T^{k}] &\\
    &&+{e_{2u}}_i^{k\ \ \ \:}                            &{A_{e}}_{i}^{k\ \ \ \:}
    &\ {\ \ \;_i^k \mathbb{R}_{+1/2}^{+1/2}} &\delta_{k+1/2}[T_{i\ \ \ \;}] &\delta_{i+1/2}[T^{k}] &\\
    %
    &&-{e_{2u}}_i^{k+1}                                &{A_{e}}_i^{k+1}
    &{_i^{k+1} \mathbb{R}_{-1/2}^{- 1/2}}   &\delta_{i-1/2}[T^{k+1}]      &\delta_{k+1/2}[T_{i}] &\\
    &&-{e_{2u}}_i^{k\ \ \ \:}                             &{A_{e}}_i^{k\ \ \ \:}
    &{\ \ \;_i^k  \mathbb{R}_{-1/2}^{+1/2}}   &\delta_{i-1/2}[T^{k\ \ \ \:}]  &\delta_{k+1/2}[T_{i}] &\\
    &&-{e_{2u}}_i^{k+1    }                             &{A_{e}}_i^{k+1}
    &{_i^{k+1} \mathbb{R}_{+1/2}^{- 1/2}}   &\delta_{i+1/2}[T^{k+1}]      &\delta_{k+1/2}[T_{i}] &\\
    &&-{e_{2u}}_i^{k\ \ \ \:}                             &{A_{e}}_i^{k\ \ \ \:}
    &{\ \ \;_i^k  \mathbb{R}_{+1/2}^{+1/2}}   &\delta_{i+1/2}[T^{k\ \ \ \:}]  &\delta_{k+1/2}[T_{i}]
    &\Bigr\}  \\
  \end{matrix}
\end{align*}
The two terms associated with the triad ${_i^k \mathbb{R}_{+1/2}^{+1/2}}$ are the same but of opposite signs,
they cancel out.
Exactly the same thing occurs for the triad ${_i^k \mathbb{R}_{-1/2}^{-1/2}}$.
The two terms associated with the triad ${_i^k \mathbb{R}_{+1/2}^{-1/2}}$ are the same but both of opposite signs and
shifted by 1 in $k$ direction.
When summing over $k$ they cancel out with the neighbouring grid points.
Exactly the same thing occurs for the triad ${_i^k \mathbb{R}_{-1/2}^{+1/2}}$ in the $i$ direction.
Therefore the sum over the domain is zero,
\ie\ the variance of the tracer is preserved by the discretisation of the skew fluxes.

\subinc{%% =================================================================================================
%% Backmatter
%% =================================================================================================

%% Bibliography
%% =================================================================================================

\phantomsection
\addcontentsline{toc}{chapter}{Bibliography}
\lohead{Bibliography}
\rehead{Bibliography}
\bibliography{../main/bibliography}

\clearpage

%% Indices
%% =================================================================================================

\phantomsection
\addcontentsline{toc}{chapter}{Indices}
\lohead{Indices}
\rehead{Indices}
\printindex[blocks]
\printindex[keys]
\printindex[modules]
\printindex[parameters]
\printindex[subroutines]

\clearpage

%% Glossary
%% =================================================================================================

%\phantomsection
%\addcontentsline{toc}{chapter}{Glossary}
%\lohead{Glossary}\rehead{Glossary}
%\printglossaries
}

\end{document}
